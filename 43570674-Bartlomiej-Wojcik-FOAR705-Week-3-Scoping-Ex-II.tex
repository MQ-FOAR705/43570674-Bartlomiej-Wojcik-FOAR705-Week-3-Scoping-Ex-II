\documentclass{article}
\usepackage[utf8]{inputenc}

\title{FOAR705 Digital Humanities \\\ Proof of Concept Scoping Exercise II \\\  Computational Analysis}

\author{Bart Wojcik}
\date{22 August 2019}

\begin{document}

\maketitle

\section{Objective}
Speed up dictionary look up of Japanese words in digitally distributed works of Japanese literature.

\subsection{Decomposition:}
\begin{enumerate}
    \item Input/get word
    \item Determine ff necessary to de-inflect word and proceed as needed
    \item Look up word
    \item Render definition 
\end{enumerate}

\subsection{Pattern recognition:}
\begin{enumerate}
    \item Get input data    
    \item Analyze input data
    \item Process analysis results
    \item Render output data
\end{enumerate}

\subsection{Algorithm design:}
This section outlines the user inputs and the tasks that the machine needs to perform at each dictionary look-up request:
\begin{enumerate}
    \item Read text under mouse cursor
    \item Determine word/phrase boundaries (as a reminder, Japanese language does not use spaces between words and the machine needs to be able to reliably pick out single and compound words as well as common set phrases in a longer string of text)
    \item \textbf{If} impossible to determine single look-up term \textbf{then} supply likely candidates and proceed based on user selection, \textbf{else} proceed
    \item Determine the part of speech (noun, na-adjective, i-adjective, verb etc)
    \item \textbf{If} inflected part of speech detected \textbf{then} de-inflect and proceed; \textbf{else} proceed
    \item Perform dictionary look-up
    \item Output Japanese pronunciation, English definition and inflection details if applicable (for example: verb in past or passive-causative form, i-adjective in past form, verb stem)
\end{enumerate}

\end{document}
